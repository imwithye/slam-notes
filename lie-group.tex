\section{李群与李代数}


\subsection{快速问答}
\begin{itemize}
    \item 李群和李代数主要解决的问题是什么?{\color{blue} 李群和李代数将一个乘法运算群转换成加法运算,通过加法运算,我们能更方便的使用微分和梯度进行寻找极值点。}
    \item 为什么我们需要使用微分?{\color{blue} 在优化问题中,往往我们不能直接找到函数的极值点,一般来说,我们通过梯度下降的方式找到函数的极值点。梯度下降要求我们对变量求导,得到梯度\(\Delta x\),在每次迭代中使用\(x+\Delta x\)作为下一步的函数值进行优化。}
    \item 为什么在SLAM优化问题中使用李群?{\color{blue} SLAM的位姿求解包含了俩个重要的矩阵,旋转矩阵\(R\)和变换矩阵\(T\)。然而这俩个矩阵对加法不封闭,他们都是乘法群。为了能获得梯度下降,我们需要将乘法群转换成加法群。}
    \item 乘法如何转换成加法?{\color{blue} 在标量运算中,我们有一个很漂亮的形式,\(f(x)=e^x\),我们有\(f(a)f(b)=f(a+b)\)。通过指数或者对数映射,我们能够将一个乘法运算转换成加法运算。因此,我们的目标是利用李群和李代数寻找矩阵的某种指数和对数映射关系。}
    \item 什么是矩阵指数?{\color{blue} 我们定义矩阵指数\(e^A\),对于标量来说,我们有\(e^0=1\),\(e^{a+b}=e^ae^b\)。在矩阵指数中,我们也应该满足类似的性质。}
\end{itemize}


\subsection{旋转矩阵的求导}
考虑任何一个旋转矩阵\(R\),我们知道旋转矩阵满足
\[
    RR^T = I
\]
考虑一个连续变换的旋转矩阵\(R(t)\),其中\(t\)表示时间参数,根据上述的方程我们可知
\[
    R(t)R(t)^T = I
\]
我们对方程两侧求导可知
\[
    R'(t)R(t)^T + R(t)R'(t)^T=0
\]
整理得到
\[
    R'(t)R(t)^T = -R(t)R'(t)^T = -(R'(t)R(t)^T)^T
\]
因此我们得到矩阵\(R'(t)R(t)^T\)是一个反对称矩阵。对于任意一个反对称矩阵来说,其对角元素为\(0\),且对角线元素为相反数。我们可以找到一个唯一对应的向量,并用运算符\(\wedge\)和\(\vee\)表示向量和矩阵的转换关系
\[
    a^\wedge = A = \begin{bmatrix}0 & -a_3 & a_2\\a_3 & 0 & -a_1\\-a_2 & a_1 & 0\end{bmatrix}, A^\vee = a
\]
所以我们可以找到一个三维向量\(\phi(t)\),使得{\color{red} \(R'(t)R(t)^T=\phi(t)^\wedge\)}。两边同时乘以\(R(t)\),我们得到
\[
    R'(t)R(t)^TR(t) = {\color{red} R'(t) = \phi(t)^\wedge R(t)}
\]
由此我们得到了在\(t\)时刻,旋转矩阵的导数与旋转矩阵的关系。
现在我们考虑{\color{blue} 泰勒定理},对于函数\(f(x)\)在\(a\)附近的值,我们有
\[
    f(x) \approx f(a) + f'(a)(x-a)
\]
我们将函数\(R(t)\),和\(a=0\)代入泰勒定理,且我们让旋转矩阵初值为单位矩阵\(I\),我们得到\(R(t)\)在\(t=0\)附近的值为
\[
    R(t) \approx R(0) + R'(0)t = I + \phi(0)^\wedge I t = I + \phi(0)^\wedge t
\]
{\color{red} 暂时未能有明确解释:由此我们可见\(\phi(t)^\wedge\)在\(t=0\)附近几乎保持不变,我们可以假设其在\(t=0\)附近时,保持为常数\(\phi_0\)。由此我们得到在\(t=0\)附近时,我们有}
\[
    R'(t) = \phi_0 R(t)
\]
这是一个很简单的微分方程,一个函数的导数等于其函数本身,我们很容易联想到指数函数,因此,我们可以得到其原函数为
\[
    R(t) = e^{\phi_0 t}
\]


\subsection{李代数的引出}
所有的旋转矩阵构成了群
\[
    SO(3) = {R \in 	\mathbb{R}}
\]


\subsection{矩阵的指数对数映射}
刚刚我们得到了在局部范围的旋转矩阵的指数形式,矩阵的指数定义如下
设矩阵\(A\)为方阵,则
\[
    e^A = \sum_{n=0}^{\infty}\frac{1}{n!}A^n
\]
然而,我们并不想引入无穷项的计算